\documentclass{article}


\usepackage{arxiv}

\usepackage[utf8]{inputenc} % allow utf-8 input
\usepackage[T1]{fontenc}    % use 8-bit T1 fonts
\usepackage{hyperref}       % hyperlinks
\usepackage{url}            % simple URL typesetting
\usepackage{booktabs}       % professional-quality tables
\usepackage{amsfonts}       % blackboard math symbols
\usepackage{nicefrac}       % compact symbols for 1/2, etc.
\usepackage{microtype}      % microtypography
\usepackage{lipsum}		% Can be removed after putting your text content
\usepackage{tikz-cd}

\title{Meat \& cancer: a critical review based on causal analysis \it{(work in progress)}}

%\date{September 9, 1985}	% Here you can change the date presented in the paper title
%\date{} 					% Or removing it

\author{
  Any volunteer?\thanks{Use footnote for providing further
    information about author (webpage, alternative
    address)---\emph{not} for acknowledging funding agencies.} \\
Any Department\\
  Anywhere\\
  %% examples of more authors
   \And
 Enrique Otero \\
  Madrid, Spain\\
  @eoteromuras \\
  \texttt{eoteromuras@gmail.com} \\
  %% \AND
  %% Coauthor \\
  %% Affiliation \\
  %% Address \\
  %% \texttt{email} \\
  %% \And
  %% Coauthor \\
  %% Affiliation \\
  %% Address \\
  %% \texttt{email} \\
  %% \And
  %% Coauthor \\
  %% Affiliation \\
  %% Address \\
  %% \texttt{email} \\
}

\begin{document}
\maketitle

\begin{abstract}
According to World Health Organization (WHO), processed meat has been declared Group 1 carcinogenic to humans. That means that according to epidemiological studies there is a convincing evidence that the agent causes cancer. However, reviewing some of the mainly referred studies with the lenses of causal inference analysis reveals possible flaws that would invalidate these conclusions. The author(s) intention is to discuss these studies with statistical rigor. By applying last accepted knowledge in the field of causal inference as diagrams and \textit{do-calculus}. With main focus on transparent exposition of health domain assumptions. And the translation of these assumptions into explicit language, diagrams and formulas. So the veracity of domain assumptions can be refuted according to domain expertise. And any conclusion derived from these assumptions being validated or invalidated on the bases of axiomatic logic and maths.

% Despite how unhealthy or unsustainable meat consumption could be, carcinogenic Group 1 statements could be based on uncomplete models, but they at least should be based on correct analysis.
\end{abstract}


% keywords can be removed
\keywords{Meat \and Cancer \and Causality}


\section{Introduction}
 In October 2015 IARC held an expert panel that considered the evidence for read and processed meats as possible human carcinogens. They classified processed meat as a Group 1 carcinogenic to humans, and red meat as Group 2A, probably carcinogenic \cite{whoint}. A summary of the final evaluations were published online in The Lancet Oncology \cite{lancet}. And the details of these conclusions were published later in a monograph in 2016 \cite{monograph}.

The consumption of processed meat was associated with small increases in the risk of cancer in the studies reviewed. In these studies, the risk generally increased with the amount of meat consumed \cite{whoint}.


In the next sections we will focus on three studies supporting IARC conclusions, and we will remark different flaws detected in them. Particularly:

\begin{itemize}

\item  In Section \ref{sec:iarc}, as starting point we will base on IARC monograph and Chan meta-analysis \cite{chan}, as it's the main reference for IARC when they conclude "each 50 gram portion of processed meat eaten daily increases the risk of colorectal cancer by 18\%". And we will remark potential problems related to heterogeneity. Though our focus will be in introducing possible problems related to not conditioning on missing confounders (Section 3), or conditioning on a collider (Section 4).
\item In Section \ref{sec:sandhu}, as example of discarding a possible confounder based on a possible wrong procedure we present Sandhu et al meta-analysis \cite{sandhu}, both referenced by Chan's \cite{chan} and IARC monograph \cite{monograph}.
\item Finally in Section \ref{sec:cross}, as an example of generating strange and wrong conclusions based on the wrong procedure of conditioning on a collider, we'll discuss a study by Cross et al \cite{cross}. This study is particularly relevant as it's the one that contributes the most to results on Chan's meta-analysis.
\end{itemize}

For this purposes we will use different causal inference techniques as causal diagrams and \textit{do-calculus}, as presented by Pearl's et al \cite{bookofwhy}.

%%%%%%%%%%%
%%%%%%%%%%%
\section{Processed Meat and Colorectal Cancer Incidence}
\label{sec:iarc}

In \cite{monograph} the IARC Working Group analyzed both "20 large [...] cohort studies [...] extended from as early as the 1990s until the 2010s", and "a large number of case-control studies (approximately 150)". In the monograph authors describe also five criteria they applied in reviewing and interpreting the available literature in order to be considered for their meta-analysis. One of these criteria, which we will focus specifically on this paper is the "Adjustment for potential confounding factors"

Regarding case-control studies, they considered that "approximately 10\% of all case–control studies reviewed were informative for the assessment of the consumption of processed meat in relation to incidence of cancer of the colorectum". Taking into account previous statement of approximately 150 control-studies, they should be 150/10 = 15 informative studies. However they say: "Six of the nine studies considered showed positive associations with cancer of the colorectum." Could this difference (9 vs. 15) be a typo?

In relation to cohort studies, authors presented conclusions from a meta-analysis including data from 10
of these studies that "reported a statistically significant dose–response association between consumption
of red meat and/or processed meat and cancer of the colorectum". More concretely, they refer to Chan et al \cite{chan}, where "dose-response relationships were expressed per increment of intake of 100 grams
per day for red and processed meat, and 50 grams per day for processed meat as in previous meta-analyses \cite{aicr}, \cite{sandhu}


\subsection{Dose-response Analysis on Processed Meat. Heterogeneity and Confounders}

In detail, in Chan's meta-analysis \cite{chan} 26 publications from 21 studies were included. Being 15 publications from 14 studies on processed meat. Results in Table~\ref{tab:table}.

\begin{table}
 \caption{Relative risks of meta-analyses of processed meat, and colorectal cancer. Chan et al. meta-analysis}
  \centering

  \begin{center}
   \begin{tabular}{||c c c||}
   \hline
   Pooled RR (95\% CI)     & n     & Heterogeneity (I2)\\ [0.5ex]
   \hline\hline
     1.18 (1.10–1.28), P-value=0.00  & 9 & 12\%, P-value=0.33     \\ [1ex]
   \hline
  \end{tabular}

  \caption*{RR – relative risk; CI – confidence interval; n – number of studies}
  \end{center}

  \label{tab:table}
\end{table}

Regarding heterogeneity, though I-squared level of 12\% may seem no significant, specially with a p-value of 0.33, critizism has been done to I-squared as an adequate measure of heterogeneity, specially when the number of studies is small. \cite{hippel}.

In relation to confounders, authors claim: "we cannot rule out residual confounding". Though "In all studies, relative risk estimates were adjusted for age and sex, and all except two adjusted for total energy intake. More than
half of the study results were adjusted for body mass index (BMI),
smoking, alcohol consumption, or physical activity, close to half
controlled for dairy food or calcium intake, social economic status,
family history of colorectal cancer, or plant food or folate intake.
In some studies, the estimates were controlled for use of nonsteroidal
anti-inflammatory drugs, fish or white meat intake". Anyway, they also say: "several
potential confounders were not included in the final statistical
models in some studies because, as the authors reported, their
inclusion in the model did not substantially modified the relative
risk estimates." However, sometimes the decision of exclude a confounder could be flawed, not only based on domain assumptions, but also in the logical procedure itself, as we will discuss on the example of Section \ref{sec:sandhu}

Authors also refer to other studies with similar conclusions, like \cite{aicr,wei2009}
Though after their conclusions they also say: "In a more recent article on the NHS and the HPFS, the
associations of red meat and processed meat and colon cancer were
attenuated after better adjustment for confounders and longer followup \cite{wei}

The biggest study analyzed in \cite{chan} regarding number of people was Cross et al. \cite{cross} with 494036 men and women. In this study adjusts were made on "Age, sex, ethnicity, BMI, smoking habits, alcohol intake, physical activity, total energy intake, fruit and vegetable intake, education level, marital status, family history of cancer"
And according to weight and results Cross's study is the one that contributes the most to RR for colorectal cancer on the consumption of processed meat on Chan's meta-analysis. So in the next section \ref{sec:cross} we'll try to proof that Cross study could be flawed because of a wrong experiment design on implicitely conditioning on a collider.



\section{Don't Discard Fiber, Vegetables, Fruits or Life-style as Possible Confounders}
\label{sec:sandhu}
In 2001 Sandhu et al published a meta-analysis on the relation between meat consumption and colorectal cancer \cite{sandhu}. Discussing about "Meat and Other Dietary and Associated Factors" they wrote: "the current
prospective epidemiological data show only a weak negative
association between vegetables and fruits consumption and risk
of colorectal cancer. Four recent studies, two randomized
trials on adenoma recurrence and two large prospective
studies on colorectal cancer found no association
among fiber, vegetables, and fruits consumption and risk of colorectal cancer."

But even if the total effect from fiber, vegetables, and fruits consumption on colorrectal cancer is negligible, matematically the direct effect even could be important. Under the assumption that consumption of vegetables was associated with consumption of meat via dietary lifestyle.

As an example, considering:

\begin{itemize}
\item X: meat
\item Y: colorrectal cancer
\item V: vegetables
\item U: some dietary lifestyle, confounder of V and X
\end{itemize}

with the following causal diagram:

\begin{tikzcd}
X \arrow{r} &Y\\
U \arrow{u} \arrow{r} &V \arrow{u}
\end{tikzcd}

If "no association among fiber, vegetables, and fruits consumption and risk of colorectal cancer"
\begin{equation}
  P(Y|do(V))=P(Y)
\end{equation}

Otherwise, according to the provided causal diagram, V blocks all back-door paths between X and Y. Thus:
\begin{equation}
P(Y|do(X))=\sum _{i=1}^{N} P(Y|X,V_i)P(V_i) \neq P (Y|X)
\end{equation}


On the basis of the previous analysis we still have no evidences to ignore direct effect of fiber, vegetables and fruits consumption with dietary lifestyle as possible confounder.

Other of the biggest studies like English et al \cite{english} included also fish as possible cause.

\section{Leukemia versus Life-style Cancers. The Collider Bias}
\label{sec:cross}

Extracted from Cross study on 2007 \cite{cross}: "Surprisingly, both leukemia and melanoma were inversely associated with processed meat intake; the inverse association for leukemia was mainly for lymphocytic leukemia (n = 534; HR = 0.70; 95\% CI = 0.52–0.93; p for trend = 0.05) and not myeloid and monocytic leukemia (n = 457; HR = 0.88; 95\% CI = 0.64–1.20; p for trend = 0.73). The associations between processed meat intake and cancer risk are summarized in Figure 2, in order of risk magnitude."

According to the described cohort follow-up and case ascertainment: "these analyses was calculated from baseline (1995–1996) until censoring at the end of 2003, or when the participant moved out of one of the eight study areas, had a cancer diagnosis, or died, whichever came first". Thus, analyzing several types of cancer as target while implicitely conditioning on not having any other cancer converts the variable "any cancer" in a collider.

Considering
\begin{itemize}
\item X: processed meat
\item Y: leukemia
\item C1: colorectal cancer
\item C: any type of cancer
\end{itemize}

and assuming the following causal diagram:

\begin{tikzcd}
X \arrow{d} \arrow{r} &Y \arrow{d}\\
C1 \arrow{r} &C
\end{tikzcd}

So conditioning on the collider C opens the path Y -> C <-C1 <-X, invalidating the results of the study. This situation is typically known as collider bias. And a famous example is the Berkson Paradox \cite{bookofwhy}.

And this could explain these strange results where cancers usually associated with dietary lifestyle as colorectal have relative risk ratios above 1 (HR=1.2), whereas leukemia shows apparently an inverse relation to processed meat, with an HR of 0.7.

\textbf{\textit{To be continued...}}

\section{Conclusions}

\textbf{\textit{To be finished...}}


\bibliographystyle{unsrt}
%\bibliography{references}  %%% Remove comment to use the external .bib file (using bibtex).
%%% and comment out the ``thebibliography'' section.


%%% Comment out this section when you \bibliography{references} is enabled.
\begin{thebibliography}{1}

\bibitem{whoint}
\newblock Q\&A on the carcinogenicity of the consumption of red meat and processed meat
\newblock https://www.who.int/features/qa/cancer-red-meat/en/
\newblock October 2015

\bibitem{lancet}
Bouvard V. et al.
\newblock Carcinogenicity of consumption of red and processed meat
%\newblock https://www.researchgate.net/profile/Veronique\_Bouvard/publication/283443910\_Carcinogenicity\_of\_consumption\_of\_red\_and\_processed\_meat/links/5ac393f4aca272a2c99910f1/Carcinogenicity-of-consumption-of-red-and-processed-meat.pdf

\bibitem{monograph}
IARC Monographs on the Evaluation of Carcinogenic Risks to Humans. Red Meat and Processed Meat. Vol 114
https://monographs.iarc.fr/wp-content/uploads/2018/06/mono114.pdf

\bibitem{chan}
Chan et al.
\newblock Red and Processed Meat and Colorectal Cancer Incidence: Meta-Analysis of Prospective Studies
https://journals.plos.org/plosone/article/file?id=10.1371/journal.pone.0020456\&type=printable

\bibitem{bookofwhy}
Pearl et al.
\newblock The Book of Why. The New Science of Cause and Effect

\bibitem{aicr}
World Cancer Research Fund/American Institute for Cancer Research. (2007)
Food, Nutrition, Physical Activity, and the Prevention of Cancer: a Global
Perspective Washington DC: AICR.

\bibitem{sandhu}
Sandhu MS, White IR, McPherson K (2001) Systematic review of the
prospective cohort studies on meat consumption and colorectal cancer risk: a
meta-analytical approach. Cancer Epidemiol Biomarkers Prev 10: 439–446

\bibitem{hippel}
Von Hippel P.
The heterogeneity statistic I2 can be biased in small meta-analyses
https://www.ncbi.nlm.nih.gov/pmc/articles/PMC4410499/
BMC Med Res Methodol. 2015; 15: 35.

\bibitem{wei}
Wei EK, Giovannucci E, Wu K, Rosner B, Fuchs CS, et al. (2004) Comparison
of risk factors for colon and rectal cancer. IntJCancer 108: 433–442.

\bibitem{wei2009}
Wei EK, Colditz GA, Giovannucci EL, Fuchs CS, Rosner BA (2009)
Cumulative risk of colon cancer up to age 70 years by risk factor status using
data from the Nurses’ Health Study. AmJEpidemiol 170: 863–872.

\bibitem{english}
English DR, MacInnis RJ, Hodge AM, Hopper JL, Haydon AM, et al. (2004)
Red meat, chicken, and fish consumption and risk of colorectal cancer. Cancer
Epidemiol Biomarkers Prev 13: 1509–1514.

\bibitem{cross}
Cross et al. 2007
A prospective study of red and processed meat intake in relation to cancer risk.

\end{thebibliography}



\end{document}
