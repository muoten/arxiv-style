\documentclass{article}


\usepackage{arxiv}

\usepackage[utf8]{inputenc} % allow utf-8 input
\usepackage[T1]{fontenc}    % use 8-bit T1 fonts
\usepackage{hyperref}       % hyperlinks
\usepackage{url}            % simple URL typesetting
\usepackage{booktabs}       % professional-quality tables
\usepackage{amsfonts}       % blackboard math symbols
\usepackage{nicefrac}       % compact symbols for 1/2, etc.
\usepackage{microtype}      % microtypography
\usepackage{lipsum}		% Can be removed after putting your text content
\usepackage{tikz-cd}

\title{Meat \& cancer: a critical review based on causal analysis \it{(work in progress)}}

%\date{September 9, 1985}	% Here you can change the date presented in the paper title
%\date{} 					% Or removing it

\author{
  Any volunteer?\thanks{Use footnote for providing further
    information about author (webpage, alternative
    address)---\emph{not} for acknowledging funding agencies.} \\
Any Department\\
  Anywhere\\
  %% examples of more authors
   \And
 Enrique Otero \\
  Madrid, Spain\\
  @eoteromuras \\
  \texttt{eoteromuras@gmail.com} \\
  %% \AND
  %% Coauthor \\
  %% Affiliation \\
  %% Address \\
  %% \texttt{email} \\
  %% \And
  %% Coauthor \\
  %% Affiliation \\
  %% Address \\
  %% \texttt{email} \\
  %% \And
  %% Coauthor \\
  %% Affiliation \\
  %% Address \\
  %% \texttt{email} \\
}

\begin{document}
\maketitle

\begin{abstract}
According to World Health Organization (WHO), processed meat has been declared Group 1 carcinogenic to humans. That means that according to epidemiological studies there is a convincing evidence that the agent causes cancer. However, reviewing some of the mainly referred studies with the lenses of causal inference analysis reveals possible flaws that would invalidate these conclusions. The author(s) intention is to discuss these studies with statistical rigor. By applying last accepted knowledge in the field of causal inference as diagrams and \textit{do-calculus}. With main focus on transparent exposition of health domain assumptions. And the translation of these assumptions into explicit language, diagrams and formulas. So the veracity of domain assumptions can be refuted according to domain expertise. And any conclusion derived from these assumptions being validated or invalidated on the bases of axiomatic logic and maths.

% Despite how unhealthy or unsustainable meat consumption could be, carcinogenic Group 1 statements could be based on uncomplete models, but they at least should be based on correct analysis.
\end{abstract}


% keywords can be removed
\keywords{Meat \and Cancer \and Causality}


\section{Introduction}
 In October 2015 IARC held an expert panel that considered the evidence for read and processed meats as possible human carcinogens. They classified processed meat as a Group 1 carcinogenic to humans, and red meat as Group 2A, probably carcinogenic \cite{whoint}. A summary of the final evaluations were published online in The Lancet Oncology \cite{lancet}. And the details of these conclusions were published later in a monograph in 2016 \cite{monograph}.

The consumption of processed meat was associated with small increases in the risk of cancer in the studies reviewed. In these studies, the risk generally increased with the amount of meat consumed \cite{whoint}.


In the next sections we will focus on three studies supporting IARC conclusions, and we will remark different flaws detected in them. Particularly:

\begin{itemize}

\item  In Section \ref{sec:chan}, as starting point we will base on IARC monograph and Chan meta-analysis \cite{chan}, as it's the main reference for IARC when they conclude "each 50 gram portion of processed meat eaten daily increases the risk of colorectal cancer by 18\%". And we will remark potential problems related to heterogeneity. Though our focus will be in introducing possible problems related to not conditioning on missing confounders (Section 3), or conditioning on a collider (Section 4).
\item In Section \ref{sec:sandhu}, as example of discarding a possible confounder based on a possible wrong procedure we present Sandhu et al meta-analysis \cite{sandhu}, both referenced by Chan's \cite{chan} and IARC monograph \cite{monograph}.
\item Finally in Section \ref{sec:cross}, as an example of generating strange and wrong conclusions based on the wrong procedure of conditioning on a collider, we'll discuss a study by Cross et al \cite{cross}. This study is particularly relevant as it's the one that contributes the most to results on Chan's meta-analysis.
\end{itemize}

For this purposes we will use different causal inference techniques as causal diagrams and \textit{do-calculus}, as presented by Pearl's et al \cite{bookofwhy}.

%%%%%%%%%%%
%%%%%%%%%%%
\section{Processed Meat and Colorectal Cancer Incidence}
\label{sec:chan}

In the IARC monograph \cite{monograph} the authors describe five criteria they applied in reviewing and interpreting the available literature in order to be considered for their meta-analysis. One of these criteria, which we will focus specifically on this paper is the "Adjustment for potential confounding factors"


Close to 20 large,
%high-quality
cohort studies were considered in the evaluation, with the results
of some studies reported in several publications.
The follow-ups of these studies extended from as
early as the 1990s until the 2010s. A large number
of case–control studies (approximately 150),
conducted across the world were reviewed for
this evaluation.

%Associations between colorectal cancer and consumption of processed meat were examined in 18 informative cohort studies and two pooled analyses. Eighteen cohort studies investigated the association between processed meat and incidence of cancer of the colorectum. Positive associations between consumption of processed meat and incidence of cancer of the colorectum were observed in 12 studies, including some of the larger studies: a Japanese cohort, the Nurses’ Health Study (NHS), the Health Professionals Follow-Up Study (HPFS), the EPIC study, the Cancer Prevention Study II (CPS-II), and the National Institutes of Health – American Association of Retired Persons (NIH-AARP) Diet and Health Study

The Working Group considered that approximately
10\% of all case–control studies reviewed
were informative for the assessment of the
consumption of processed meat in relation to
incidence of cancer of the colorectum. So they should be 150/10 = 15 studies. Anyway, just after that they say: "Six of the nine studies considered showed positive associations with cancer of the colorectum.". Is this a typo?

Moreover they present conclusions from a meta-analysis including data from 10
cohort studies that reported a statistically significant
dose–response association between consumption
of red meat and/or processed meat and cancer of
the colorectum. More concretely, they refer to Chan et al \cite{chan}
%"[...] the relative risks for the highest compared with the lowest intake [...] for anincrease of 100 g/day of red meat was 1.17 (95\% CI, 1.05–1.31; 8 studies) for colorectal cancer [...] For processed meats, the relative risk for the highest compared with the lowest intake was 1.17 (95\% CI, 1.09–1.25; I2 = 6\%; 13 studies) for colorectal cancer" "The relative risks for an increase of 50 g/day were 1.18 (95\% CI, 1.10–1.28; I2 = 12\%; 9 studies) for colorectal cancer, ".



In \cite{chan} they "conducted meta-analyses
for red and processed meats, combined and separately, using the
description of the meat items given in the articles. In highest versus
lowest meta-analyses (the comparison of the highest intake level to
the lowest intake level), the relative risk (RR) estimate from each
study was weighted by the inverse of the variance to calculate
summary relative risks (RR) and 95\% confidence intervals (CI). In
linear dose-response meta-analyses, we pooled the relative risk
estimates per unit of intake increase (with its standard error)
reported in the studies, or computed by us from the categorical
data using generalized least-squares for trend estimation"

Furthermore, "Dose-response relationships were expressed per increment of intake of 100 grams
per day for red and processed meat, and 50 grams per day for processed meat as in previous meta-analyses \cite{aicr}, \cite{sandhu}

And "To assess heterogeneity, we computed the Cochran Q test and I2
statistic. Sources of heterogeneity were explored in stratified
analysis and by linear meta-regression, with gender, geographic
area, year of publication, length of follow-up, and adjustment for
confounders as potential explanatory factors".




\subsection{Dose-response Analysis on Processed Meat. And Confounders}

In detail, in \cite{chan} 26 publications from 21 studies were included in the dose response meta-analysis of cancer incidence. Being 15 publications from 14 studies on processed meat. Results were: Pooled RR (95\% CI)=1.18 (1.10–1.28), P-value=0.00, n=9, Heterogeneity (I2)=12\%, P-value=0.33. Though I-squared level of 12\% may seem no significant, specially with a p-value of 0.33, critizism has been done to I-2 as measure of heterogeneity, specially when the number of studies is small \cite{hippel}.

Chan et al also refer to other studies with similar conclusions, like \cite{aicr,wei2009}
Though after their conclusions authors also say: "In a more recent article on the NHS and the HPFS, the
associations of red meat and processed meat and colon cancer were
attenuated after better adjustment for confounders and longer followup \cite{wei}


Although we cannot rule out residual confounding, most studies
included in the meta-analyses adjusted results by smoking, alcohol
consumption, BMI and physical activity in addition to age, sex and energy; in several
cohort studies the multivariate adjusted models also included folate
intake, and other studies additionally adjusted for
aspirin or other anti-inflammatory drug use. Several
potential confounders were not included in the final statistical
models in some studies because, as the authors reported, their
inclusion in the model did not substantially modified the relative
risk estimates.

Concretely: "In all studies, relative risk estimates were adjusted for age and
sex, and all except two adjusted for total energy intake. More than
half of the study results were adjusted for body mass index (BMI),
smoking, alcohol consumption, or physical activity, close to half
controlled for dairy food or calcium intake, social economic status,
family history of colorectal cancer, or plant food or folate intake.
In some studies, the estimates were controlled for use of nonsteroidal
anti-inflammatory drugs, fish or white meat intake".

And for instance: "Stratified analysis did not suggest any difference across gender. The
association between red meat and colon cancer tended to be stronger
in European studies (RR for 100 g/day increase = 1.29, 95\% CI =
1.0821.54) (3 studies, 1307 cases) compared to the North American
(RR for 100 g/day increase = 1.11, 95\% CI = 0.8621.44) (4 studies, 1476
cases) and Asia-Pacific studies (RR for100 g/day increase = 0.94, 95\% CI =
0.6921.27, P = 0.67) (3 studies, 732 cases)."

The biggest study analyzed in \cite{chan} regarding number of people was \cite{cross} with 494036 men and women. In this study adjusts were made on "Age, sex, ethnicity, BMI, smoking habits, alcohol intake, physical activity, total energy intake, fruit and vegetable intake, education level, marital status, family history of cancer"
This study according to weight and results is the one that contributes the most to RR for colorectal cancer on the consumption of processed meat on Chan's meta-analysis. In the next section we'll try to proof that Cross study could be flawed because of a wrong assumption and wrong experiment design on adjusting on too much variables.



\section{Don't Discard Fiber, Vegetables, Fruits or Life-style as Possible Confounder}
\label{sec:sandhu}
In 2001 Sandhu et al published a meta-analysis on the relation between meat consumption and colorectal cancer \cite{sandhu}. When discussing "Meat and Other Dietary and Associated Factors" they say: "However, the current
prospective epidemiological data show only a weak negative
association between vegetables and fruits consumption and risk
of colorectal cancer (4, 7). Four recent studies, two randomized
trials on adenoma recurrence (57, 58) and two large prospective
studies on colorectal cancer (59, 60) found no association
among fiber, vegetables, and fruits consumption and risk of colorectal cancer. The two prospective studies based on the
Nurses’ Health Study (59) and a combined analysis of the
Nurses’ Heath Study and the Health Professionals’ Follow-up
Study (60) both adjusted for red meat intake when ascertaining
the effect of fiber and vegetables and fruits consumption on
colorectal cancer risk, respectively. The multivariate estimates
did not materially differ from the unadjusted estimates".

But even if the total effect from fiber, vegetables, and fruits consumption on colorrectal cancer is negligible, the direct effect could be important.

As an example, considering:

\begin{itemize}
\item X: meat
\item Y: colorrectal cancer
\item V: vegetables
\item U: some dietary lifestyle, confounder of V and X
\end{itemize}

with the following causal diagram:

\begin{tikzcd}
X \arrow{r} &Y\\
U \arrow{u} \arrow{r} &V \arrow{u}
\end{tikzcd}

If "no association among fiber, vegetables, and fruits consumption and risk of colorectal cancer"
\begin{equation}
  P(Y|do(V))=P(Y)
\end{equation}

Otherwise, according to the provided causal diagram:
\begin{equation}
P(Y|do(X))=\sum _{i=1}^{N} P(Y|X,V_i)P(V_i) \neq P (Y|X)
\end{equation}


So we shouldn't discard fiber, vegetables and fruits consumption as possible confounders on the basis of the previous analysis.

Other of the biggest studies like English et al \cite{english} included also fish as possible cause.

Moreover, considering dietary lifestyle as a confounder helps to understand some strange results obtained in studies like \cite{cross} that we will discuss in the next section.

\section{Leukemia versus Life-style Cancers. The Berkson Bias}
\label{sec:cross}

Extracted from Cross study on 2007 \cite{cross}: "Surprisingly, both leukemia and melanoma were inversely associated with processed meat intake; the inverse association for leukemia was mainly for lymphocytic leukemia (n = 534; HR = 0.70; 95\% CI = 0.52–0.93; p for trend = 0.05) and not myeloid and monocytic leukemia (n = 457; HR = 0.88; 95\% CI = 0.64–1.20; p for trend = 0.73). The associations between processed meat intake and cancer risk are summarized in Figure 2, in order of risk magnitude."

\textbf{\textit{To be continued...}}

\section{Conclusions}

\textbf{\textit{To be finished...}}


\bibliographystyle{unsrt}
%\bibliography{references}  %%% Remove comment to use the external .bib file (using bibtex).
%%% and comment out the ``thebibliography'' section.


%%% Comment out this section when you \bibliography{references} is enabled.
\begin{thebibliography}{1}

\bibitem{whoint}
\newblock Q\&A on the carcinogenicity of the consumption of red meat and processed meat
\newblock https://www.who.int/features/qa/cancer-red-meat/en/
\newblock October 2015

\bibitem{lancet}
Bouvard V. et al.
\newblock Carcinogenicity of consumption of red and processed meat
%\newblock https://www.researchgate.net/profile/Veronique\_Bouvard/publication/283443910\_Carcinogenicity\_of\_consumption\_of\_red\_and\_processed\_meat/links/5ac393f4aca272a2c99910f1/Carcinogenicity-of-consumption-of-red-and-processed-meat.pdf

\bibitem{monograph}
IARC Monographs on the Evaluation of Carcinogenic Risks to Humans. Red Meat and Processed Meat. Vol 114
https://monographs.iarc.fr/wp-content/uploads/2018/06/mono114.pdf

\bibitem{chan}
Chan et al.
\newblock Red and Processed Meat and Colorectal Cancer Incidence: Meta-Analysis of Prospective Studies
https://journals.plos.org/plosone/article/file?id=10.1371/journal.pone.0020456\&type=printable

\bibitem{bookofwhy}
Pearl et al.
\newblock The Book of Why. The New Science of Cause and Effect

\bibitem{aicr}
World Cancer Research Fund/American Institute for Cancer Research. (2007)
Food, Nutrition, Physical Activity, and the Prevention of Cancer: a Global
Perspective Washington DC: AICR.

\bibitem{sandhu}
Sandhu MS, White IR, McPherson K (2001) Systematic review of the
prospective cohort studies on meat consumption and colorectal cancer risk: a
meta-analytical approach. Cancer Epidemiol Biomarkers Prev 10: 439–446

\bibitem{hippel}
Von Hippel P.
The heterogeneity statistic I2 can be biased in small meta-analyses
https://www.ncbi.nlm.nih.gov/pmc/articles/PMC4410499/
BMC Med Res Methodol. 2015; 15: 35.

\bibitem{wei}
Wei EK, Giovannucci E, Wu K, Rosner B, Fuchs CS, et al. (2004) Comparison
of risk factors for colon and rectal cancer. IntJCancer 108: 433–442.

\bibitem{wei2009}
Wei EK, Colditz GA, Giovannucci EL, Fuchs CS, Rosner BA (2009)
Cumulative risk of colon cancer up to age 70 years by risk factor status using
data from the Nurses’ Health Study. AmJEpidemiol 170: 863–872.

\bibitem{english}
English DR, MacInnis RJ, Hodge AM, Hopper JL, Haydon AM, et al. (2004)
Red meat, chicken, and fish consumption and risk of colorectal cancer. Cancer
Epidemiol Biomarkers Prev 13: 1509–1514.

\bibitem{cross}
Cross et al. 2007
A prospective study of red and processed meat intake in relation to cancer risk.

\end{thebibliography}



\end{document}
